%% Generated by Sphinx.
\def\sphinxdocclass{report}
\documentclass[letterpaper,10pt,english]{sphinxmanual}
\ifdefined\pdfpxdimen
   \let\sphinxpxdimen\pdfpxdimen\else\newdimen\sphinxpxdimen
\fi \sphinxpxdimen=.75bp\relax
\ifdefined\pdfimageresolution
    \pdfimageresolution= \numexpr \dimexpr1in\relax/\sphinxpxdimen\relax
\fi
%% let collapsible pdf bookmarks panel have high depth per default
\PassOptionsToPackage{bookmarksdepth=5}{hyperref}

\PassOptionsToPackage{booktabs}{sphinx}
\PassOptionsToPackage{colorrows}{sphinx}

\PassOptionsToPackage{warn}{textcomp}
\usepackage[utf8]{inputenc}
\ifdefined\DeclareUnicodeCharacter
% support both utf8 and utf8x syntaxes
  \ifdefined\DeclareUnicodeCharacterAsOptional
    \def\sphinxDUC#1{\DeclareUnicodeCharacter{"#1}}
  \else
    \let\sphinxDUC\DeclareUnicodeCharacter
  \fi
  \sphinxDUC{00A0}{\nobreakspace}
  \sphinxDUC{2500}{\sphinxunichar{2500}}
  \sphinxDUC{2502}{\sphinxunichar{2502}}
  \sphinxDUC{2514}{\sphinxunichar{2514}}
  \sphinxDUC{251C}{\sphinxunichar{251C}}
  \sphinxDUC{2572}{\textbackslash}
\fi
\usepackage{cmap}
\usepackage[T1]{fontenc}
\usepackage{amsmath,amssymb,amstext}
\usepackage{babel}



\usepackage{tgtermes}
\usepackage{tgheros}
\renewcommand{\ttdefault}{txtt}



\usepackage[Bjarne]{fncychap}
\usepackage{sphinx}

\fvset{fontsize=auto}
\usepackage{geometry}


% Include hyperref last.
\usepackage{hyperref}
% Fix anchor placement for figures with captions.
\usepackage{hypcap}% it must be loaded after hyperref.
% Set up styles of URL: it should be placed after hyperref.
\urlstyle{same}


\usepackage{sphinxmessages}




\title{FMD\_QNC\_amps\_filled}
\date{Dec 18, 2024}
\release{0.1}
\author{oliver munz}
\newcommand{\sphinxlogo}{\vbox{}}
\renewcommand{\releasename}{Release}
\makeindex
\begin{document}

\ifdefined\shorthandoff
  \ifnum\catcode`\=\string=\active\shorthandoff{=}\fi
  \ifnum\catcode`\"=\active\shorthandoff{"}\fi
\fi

\pagestyle{empty}
\sphinxmaketitle
\pagestyle{plain}
\sphinxtableofcontents
\pagestyle{normal}
\phantomsection\label{\detokenize{index::doc}}


\begin{sphinxadmonition}{warning}{Warning:}
\sphinxAtStartPar
this chip is designed without correct LVS. only parts are really verified…
\end{sphinxadmonition}

\noindent{\hspace*{\fill}\sphinxincludegraphics{{chip}.png}\hspace*{\fill}}

\sphinxAtStartPar


\sphinxstepscope


\chapter{specifications}
\label{\detokenize{specification:specifications}}\label{\detokenize{specification::doc}}
\sphinxAtStartPar
there are 3 different amplifiers, a voltage\sphinxhyphen{}regulator and some experiments
\begin{enumerate}
\sphinxsetlistlabels{\arabic}{enumi}{enumii}{}{.}%
\item {} \begin{description}
\sphinxlineitem{single ended lowpower amplifier}\begin{itemize}
\item {} 
\sphinxAtStartPar
3.3V power supply

\item {} 
\sphinxAtStartPar
\textless{} 3.5mA current including powering of a laser\sphinxhyphen{}diode

\item {} 
\sphinxAtStartPar
\textgreater{} 30dB gain at 1GHz

\item {} 
\sphinxAtStartPar
0..60°C

\item {} 
\sphinxAtStartPar
low\sphinxhyphen{}pass corner programmable via capacitor

\end{itemize}

\end{description}

\item {} \begin{description}
\sphinxlineitem{dual input differential lowpower amplifier}\begin{itemize}
\item {} 
\sphinxAtStartPar
3.3V power supply

\item {} 
\sphinxAtStartPar
\textless{} 3.5mA current including powering of a laser\sphinxhyphen{}diode

\item {} 
\sphinxAtStartPar
\textgreater{} 10dB gain at 1GHz

\item {} 
\sphinxAtStartPar
0..60°C

\end{itemize}

\end{description}

\item {} \begin{description}
\sphinxlineitem{differential lowpower amplifier}\begin{itemize}
\item {} 
\sphinxAtStartPar
3.3V power supply

\item {} 
\sphinxAtStartPar
\textless{} 3.5mA current including powering of a laser\sphinxhyphen{}diode

\item {} 
\sphinxAtStartPar
0dB gain at 1GHz

\item {} 
\sphinxAtStartPar
0..60°C

\end{itemize}

\end{description}

\item {} \begin{description}
\sphinxlineitem{shunt\sphinxhyphen{}regulator 3.3V}\begin{itemize}
\item {} 
\sphinxAtStartPar
max. 50mA

\item {} 
\sphinxAtStartPar
delta tc = 0 @ 10..30°C

\item {} 
\sphinxAtStartPar
trimmable

\item {} 
\sphinxAtStartPar
possibility for lowpass filter

\end{itemize}

\end{description}

\item {} \begin{description}
\sphinxlineitem{test\sphinxhyphen{}structure for RPPD matching}\begin{itemize}
\item {} 
\sphinxAtStartPar
see differences in x and y build resistors, does it really matter?

\item {} 
\sphinxAtStartPar
see differences in middle and edge resistors, do we need dummy resistors?

\end{itemize}

\end{description}

\item {} \begin{description}
\sphinxlineitem{a sg13g2 transistor}\begin{itemize}
\item {} 
\sphinxAtStartPar
if nothing else is working, we get at least a transistor :)

\end{itemize}

\end{description}

\item {} \begin{description}
\sphinxlineitem{open, short \& load}\begin{itemize}
\item {} 
\sphinxAtStartPar
to calibrate the pads and connections out, and measure the impedances of the amplifiers

\end{itemize}

\end{description}

\end{enumerate}


\section{pinout}
\label{\detokenize{specification:pinout}}
\noindent{\hspace*{\fill}\sphinxincludegraphics{{pinout}.eps}\hspace*{\fill}}


\section{signals}
\label{\detokenize{specification:signals}}

\begin{savenotes}\sphinxattablestart
\sphinxthistablewithglobalstyle
\centering
\begin{tabulary}{\linewidth}[t]{TT}
\sphinxtoprule
\sphinxstyletheadfamily 
\sphinxAtStartPar
pin name
&\sphinxstyletheadfamily 
\sphinxAtStartPar
whats it
\\
\sphinxmidrule
\sphinxtableatstartofbodyhook\sphinxstyletheadfamily 
\sphinxAtStartPar
GND
&
\sphinxAtStartPar
supply and signal ground
\\
\sphinxhline\sphinxstyletheadfamily 
\sphinxAtStartPar
SH\_GND
&
\sphinxAtStartPar
shunt\sphinxhyphen{}anode, connect to GND, optional current measurement
\\
\sphinxhline\sphinxstyletheadfamily 
\sphinxAtStartPar
3.3V
&
\sphinxAtStartPar
positive supply \& shunt regulator cathode
\\
\sphinxhline\sphinxstyletheadfamily 
\sphinxAtStartPar
ref
&
\sphinxAtStartPar
reference output \& input. optional low\sphinxhyphen{}pass capacitor to GND
\\
\sphinxhline\sphinxstyletheadfamily 
\sphinxAtStartPar
adj
&
\sphinxAtStartPar
adjust the band\sphinxhyphen{}gap\sphinxhyphen{}reference via resistor to 3.3V or GND
\\
\sphinxhline\sphinxstyletheadfamily 
\sphinxAtStartPar
in
&
\sphinxAtStartPar
amplifier input
\\
\sphinxhline\sphinxstyletheadfamily 
\sphinxAtStartPar
out
&
\sphinxAtStartPar
amplifier out
\\
\sphinxhline\sphinxstyletheadfamily 
\sphinxAtStartPar
d\sphinxhyphen{}
&
\sphinxAtStartPar
differential amplifier inverting input
\\
\sphinxhline\sphinxstyletheadfamily 
\sphinxAtStartPar
d+
&
\sphinxAtStartPar
differential amplifier non\sphinxhyphen{}inverting input
\\
\sphinxhline\sphinxstyletheadfamily 
\sphinxAtStartPar
dout
&
\sphinxAtStartPar
differential amplifier laser diode output
\\
\sphinxhline\sphinxstyletheadfamily 
\sphinxAtStartPar
dd\sphinxhyphen{}
&
\sphinxAtStartPar
double differential amplifier inverting input. connected together over two 50Omega
\\
\sphinxhline\sphinxstyletheadfamily 
\sphinxAtStartPar
dd+
&
\sphinxAtStartPar
double differential amplifier non\sphinxhyphen{}inverting input. connected together over two 50Omega
\\
\sphinxhline\sphinxstyletheadfamily 
\sphinxAtStartPar
ddout
&
\sphinxAtStartPar
double differential amplifier laser diode output
\\
\sphinxbottomrule
\end{tabulary}
\sphinxtableafterendhook\par
\sphinxattableend\end{savenotes}

\sphinxstepscope


\chapter{amplifier}
\label{\detokenize{amplifiers:amplifier}}\label{\detokenize{amplifiers::doc}}

\section{schematic}
\label{\detokenize{amplifiers:schematic}}
\noindent{\hspace*{\fill}\sphinxincludegraphics{{single}.eps}\hspace*{\fill}}

\sphinxAtStartPar
this single input amplifier ist DC\sphinxhyphen{}coupled and GND referenced. it drives a laser diode over Q1. its bias current is measured via Q8 and regulated from a norton amplifier build from Q11, Q7 and Q9. the low\sphinxhyphen{}pass\sphinxhyphen{}corner is set by R12 and C2 (that is also available over test\sphinxhyphen{}pads inside the power\sphinxhyphen{}ring of the chip).
the RF\sphinxhyphen{}path goes over a base\sphinxhyphen{}circuit Q6 to Q10 and Q1. R3 is used to set the gain.
Q2’s job is to improve the temperature dependence of the bias\sphinxhyphen{}current.


\section{layout}
\label{\detokenize{amplifiers:layout}}
\noindent{\hspace*{\fill}\sphinxincludegraphics{{single_layout}.png}\hspace*{\fill}}

\sphinxAtStartPar
the RPPD resistors are layout that way, because i had problems using LVS with resistors and was hoping it works with the simples shape. but in the end, i didn’t manage to make LVS work anyway.


\chapter{differential amplifier}
\label{\detokenize{amplifiers:differential-amplifier}}

\section{schematic}
\label{\detokenize{amplifiers:id1}}
\noindent{\hspace*{\fill}\sphinxincludegraphics{{d0}.eps}\hspace*{\fill}}

\sphinxAtStartPar
a simple voltage feedback (R9, R10 to R20, R21) differential amplifier, without common\sphinxhyphen{}mode\sphinxhyphen{}regulator. the output is converted over a current differencing amplifier (Q5, Q9 and Q14, Q3, Q6, Q11) to a single\sphinxhyphen{}ended signal.


\section{layout}
\label{\detokenize{amplifiers:id2}}
\noindent{\hspace*{\fill}\sphinxincludegraphics{{diff_layout}.png}\hspace*{\fill}}


\chapter{dual input differential amplifier}
\label{\detokenize{amplifiers:dual-input-differential-amplifier}}

\section{schematic}
\label{\detokenize{amplifiers:id3}}
\noindent{\hspace*{\fill}\sphinxincludegraphics{{d15}.eps}\hspace*{\fill}}

\sphinxAtStartPar
a simple current feedback (R10, R12) differential amplifier, without common\sphinxhyphen{}mode\sphinxhyphen{}regulator. the output is converted over a current differencing amplifier (Q12, Q10, Q11) to a single\sphinxhyphen{}ended signal.

\sphinxAtStartPar
the input ip is connected over two 50Omega resistors to both dd+ pads. the input in is connected over two 50Omega resistors to both dd\sphinxhyphen{} pads. its thought for compensation circuits that compensate capacitive coupled signals at the input.


\section{layout}
\label{\detokenize{amplifiers:id4}}
\noindent{\hspace*{\fill}\sphinxincludegraphics{{ddiff_layout}.png}\hspace*{\fill}}

\sphinxAtStartPar
on the left side of the amplifier the for 50Omega resistors are connected to the input\sphinxhyphen{}microstriplines.


\chapter{amplifier simulations}
\label{\detokenize{amplifiers:amplifier-simulations}}
\sphinxAtStartPar
\sphinxcode{\sphinxupquote{PDF with Xyce simulation}}

\sphinxstepscope


\chapter{shunt regulator 3.3V}
\label{\detokenize{regulator:shunt-regulator-3-3v}}\label{\detokenize{regulator::doc}}

\section{schematic}
\label{\detokenize{regulator:schematic}}
\noindent{\hspace*{\fill}\sphinxincludegraphics{{shunt}.eps}\hspace*{\fill}}

\sphinxAtStartPar
this is a shunt\sphinxhyphen{}regulator for 3.3V and max. 50mA using a simple band\sphinxhyphen{}gap\sphinxhyphen{}reference. the reference uses MOSFET and BjT temperature coefficients, and needs only one BjT.


\section{layout}
\label{\detokenize{regulator:layout}}
\noindent{\hspace*{\fill}\sphinxincludegraphics{{reg_layout}.png}\hspace*{\fill}}


\section{options}
\label{\detokenize{regulator:options}}
\sphinxAtStartPar
there is the option to filter the reference output via a capacitor from the ref\sphinxhyphen{}pad to GND.

\sphinxAtStartPar
its also possible to adjust the temperature\sphinxhyphen{}turning\sphinxhyphen{}point via the adj\sphinxhyphen{}pad and the voltage via the ref pad.


\section{simulation}
\label{\detokenize{regulator:simulation}}
\sphinxAtStartPar
\sphinxcode{\sphinxupquote{PDF with Xyce simulation}}

\sphinxstepscope


\chapter{high voltage OTA}
\label{\detokenize{ota:high-voltage-ota}}\label{\detokenize{ota::doc}}
\sphinxAtStartPar
1.1. the OTA
the circuit should be used with iHPs PNP\sphinxhyphen{}device pnpMPA, so it should be working from VSS. this called for an PMOS input. the design is a simplified version of
\sphinxcode{\sphinxupquote{The Recycling Folded Cascode: A General
Enhancement of the Folded Cascode Amplifier}}

\sphinxAtStartPar
because of the used CMOS\sphinxhyphen{}process the PMOS transistors could have an isolated bulk, without special effort. the simulations showed that the isolated versions had a bigger gain, but a smaller common\sphinxhyphen{}mode\sphinxhyphen{}range, and i preferred the later.
the bias\sphinxhyphen{}current is programmable an so also the band\sphinxhyphen{}width. the bias\sphinxhyphen{}voltages should allow wide\sphinxhyphen{}swing output voltages.
in the space the circuit uses are 3 MiM\sphinxhyphen{}capacitors placed. they are intended to use for frequency\sphinxhyphen{}compensation or as power\sphinxhyphen{}rail decoupling.


\section{OTA}
\label{\detokenize{ota:ota}}
\noindent{\hspace*{\fill}\sphinxincludegraphics{{OTA3C}.eps}\hspace*{\fill}}


\section{bias generator}
\label{\detokenize{ota:bias-generator}}
\noindent{\hspace*{\fill}\sphinxincludegraphics{{OTA33_BiAS}.eps}\hspace*{\fill}}


\section{layout}
\label{\detokenize{ota:layout}}
\noindent{\hspace*{\fill}\sphinxincludegraphics{{OTA_layout}.png}\hspace*{\fill}}

\sphinxAtStartPar



\section{simulations}
\label{\detokenize{ota:simulations}}
\sphinxAtStartPar
using different bias\sphinxhyphen{}currents of 1, 3 and 10µA a few simulations are printed into a PDF that allow to see the gain, common\sphinxhyphen{}mode\sphinxhyphen{}range, bandwidth and a slew\sphinxhyphen{}rate:

\sphinxAtStartPar
\sphinxcode{\sphinxupquote{PDF with Xyce simulations .ac .dc .trans}}

\sphinxAtStartPar
the schematics of this simulations is \sphinxcode{\sphinxupquote{Xschem document}}


\section{ETHZ feedback}
\label{\detokenize{ota:ethz-feedback}}
\sphinxAtStartPar
its a bit stupid to design OTAs that fit in a square, if there is no such space\sphinxhyphen{}requirement. the layout should be changed to minimize the conductor\sphinxhyphen{}length of the signals between the differential\sphinxhyphen{}stage and the current\sphinxhyphen{}mirrors.

\sphinxstepscope


\chapter{experiments}
\label{\detokenize{experiments:experiments}}\label{\detokenize{experiments::doc}}

\section{RPPD}
\label{\detokenize{experiments:rppd}}
\noindent{\hspace*{\fill}\sphinxincludegraphics{{exp_layout}.png}\hspace*{\fill}}

\sphinxAtStartPar
there are 6 resistors available to measure:
3 in X and Y orientations
and both versions as dummy and normal resistors


\section{sg13g2}
\label{\detokenize{experiments:sg13g2}}
\sphinxAtStartPar
there is a fast BjT for measurements.


\section{calibration kit}
\label{\detokenize{experiments:calibration-kit}}
\sphinxAtStartPar
to measure the amplifier impedances and to see the effect of pads and wire\sphinxhyphen{}bonding, there are an open, short and load calkit.

\sphinxstepscope


\chapter{design data and design process description}
\label{\detokenize{designdata:design-data-and-design-process-description}}\label{\detokenize{designdata::doc}}

\section{iHP 130nm BiCMOS process sg13g2}
\label{\detokenize{designdata:ihp-130nm-bicmos-process-sg13g2}}
\sphinxAtStartPar
the process is useable via iHPs openPDK:

\sphinxAtStartPar
source
\sphinxurl{https://github.com/IHP-GmbH/IHP-Open-PDK}

\sphinxAtStartPar
documentation
\sphinxurl{https://ihp-open-pdk-docs.readthedocs.io/en/latest/index.html}

\sphinxAtStartPar
open\sphinxhyphen{}source runs:
\sphinxurl{https://www.ihp-microelectronics.com/services/research-and-prototyping-service/fast-design-enablement/open-source-pdk}


\section{Xschem schematics:}
\label{\detokenize{designdata:xschem-schematics}}

\begin{savenotes}\sphinxattablestart
\sphinxthistablewithglobalstyle
\centering
\begin{tabulary}{\linewidth}[t]{TT}
\sphinxtoprule
\sphinxstyletheadfamily 
\sphinxAtStartPar
folder
&\sphinxstyletheadfamily 
\sphinxAtStartPar
for
\\
\sphinxmidrule
\sphinxtableatstartofbodyhook\sphinxstyletheadfamily 
\sphinxAtStartPar
design\_data/xschem/OTA
&
\sphinxAtStartPar
design and simulation of the OTA
\\
\sphinxhline\sphinxstyletheadfamily 
\sphinxAtStartPar
design\_data/xschem/amplifiers
&
\sphinxAtStartPar
simulations of the three amplifiers
\\
\sphinxhline\sphinxstyletheadfamily 
\sphinxAtStartPar
design\_data/xschem/shunt\_regulator
&
\sphinxAtStartPar
simulations of the voltage regulators
\\
\sphinxhline\sphinxstyletheadfamily 
\sphinxAtStartPar
design\_data/xschem/symbol
&
\sphinxAtStartPar
symbols for Xschem schematics
\\
\sphinxbottomrule
\end{tabulary}
\sphinxtableafterendhook\par
\sphinxattableend\end{savenotes}


\section{KLayout .GDS:}
\label{\detokenize{designdata:klayout-gds}}

\begin{savenotes}\sphinxattablestart
\sphinxthistablewithglobalstyle
\centering
\begin{tabulary}{\linewidth}[t]{TT}
\sphinxtoprule
\sphinxstyletheadfamily 
\sphinxAtStartPar
file
&\sphinxstyletheadfamily 
\sphinxAtStartPar
for
\\
\sphinxmidrule
\sphinxtableatstartofbodyhook\sphinxstyletheadfamily 
\sphinxAtStartPar
design\_data/FMD\_QNC\_amps\_filled.gds
&
\sphinxAtStartPar
layout of the chip
\\
\sphinxbottomrule
\end{tabulary}
\sphinxtableafterendhook\par
\sphinxattableend\end{savenotes}

\sphinxstepscope


\chapter{validation of the circuits}
\label{\detokenize{validation:validation-of-the-circuits}}\label{\detokenize{validation::doc}}
\sphinxAtStartPar
asap \& subito…

\begin{sphinxcontents}
\sphinxstylecontentstitle{Table of Contents}
\begin{itemize}
\item {} 
\sphinxAtStartPar
\phantomsection\label{\detokenize{index:id1}}{\hyperref[\detokenize{index:three-sige-hbt-amplifiers}]{\sphinxcrossref{Three SiGe HBT amplifiers}}}

\end{itemize}
\end{sphinxcontents}



\renewcommand{\indexname}{Index}
\printindex
\end{document}